\documentclass[12pt]{article}
\usepackage{Sweave}
\usepackage{natbib}
\usepackage{hyperref}
\usepackage[utf8]{inputenc}

\bibliographystyle{plainnat}

\Sconcordance{concordance:patchDVI.tex:/Users/murdoch/svn/MyR/patchDVI/vignettes/patchDVI.Rnw:1 %
82 1 1 2 4 0 1 2 1 1 1 2 4 0 1 2 2 1 1 2 4 0 1 2 7 1 1 2 1 0 1 1 3 0 2 %
2 1 0 1 1 3 0 1 2 347 1 1 2 4 0 1 2 222 1 1 2 1 0 1 1 3 0 1 2 16 1}


\newcommand{\patchDVI}{\patchDVIn\ }
\newcommand{\patchDVIn}{\textbf{patchDVI}}
\newcommand{\file}[1]{\texttt{#1}}

%\VignetteIndexEntry{The patchDVI package}
%\VignetteKeyword{Sweave}
%\VignetteKeyword{pdf}

\title{The \patchDVI package}
\author{Duncan Murdoch}

\begin{document}

\maketitle

\abstract{
The \patchDVI package works with Sweave \citep{Leisch02}  and knitr \citep{Xie2013}
and document previewers to facilitate editing:  it modifies the links
that \LaTeX\ puts into the output so that they refer to the original
source.  It also includes a few project management functions to 
make large multi-file documents easier to handle.}

\newpage

\tableofcontents

\newpage

\section{Introduction}

Most implementations of \LaTeX\ allow source references to be emitted, so
that previewers of the \file{.dvi} or \file{.pdf} output file can
link back to the original source line.  This has been a feature of the
\texttt{yap} previewer for \file{.dvi} files in MikTeX \citep{Schenk10}
for many
years.  Support for source references has appeared more recently
for \file{.pdf}
output, first in \file{pdfsync}.  Most recently
Synctex \citep{Laurens08} links have been
implemented in \texttt{pdflatex} and other \LaTeX\ processors.

Unfortunately for knitr/Sweave users, these links point to the \file{.tex}
source that was processed, which is not the true source code in the
knitr/Sweave \file{.Rnw} or \file{.Snw} or other input file.  (I will
refer to all of these as \file{.Rnw} files.)
Clicking on ``go to source''
in a previewer will jump to the \file{.tex} file; changes made there
will be lost the next time the \file{.Rnw} input is processed.

I wrote the \patchDVI package to address this problem.  It works as
follows.  If the knitr/Sweave file is processed with the option
\texttt{concordance=TRUE}, knitr or Sweave will output a record of the
concordance between the lines in the input file and the output file.
When the file is processed by \LaTeX, this information is embedded in
the output.  (Details of the embedding are described in sections \ref{sec:concordance}
to \ref{sec:synctex} below.)  After
producing the \file{.dvi} or \file{.pdf} file, a \patchDVI
function is run to read the concordance information and to patch the
source reference information produced by \LaTeX.  Once this has been
done, previewers will see references to the original \file{.Rnw} source
rather than the \LaTeX\ intermediate file.

Besides the technical details mentioned above, 
this paper describes the history of \patchDVI in Section \ref{sec:history} and
in section \ref{sec:make} some
project management functions.  It concludes with a short discussion.

\section{Quick Start Instructions}\label{sec:quickstart}

There are several ways to make use of \patchDVIn.  This section describes some common ones.

In all cases the package needs to be installed first; the current release
is on CRAN and can be installed using
\begin{Schunk}
\begin{Sinput}
> install.packages("patchDVI")
\end{Sinput}
\end{Schunk}
Source code is maintained on R-forge, and the latest development version
can be installed using
\begin{Schunk}
\begin{Sinput}
> install.packages("patchDVI", repos="http://R-forge.r-project.org")
\end{Sinput}
\end{Schunk}

The document also needs to have an option set to produce the ``concordances'' (links between the 
\file{.Rnw} source and the \file{.tex} output of knitr/Sweave).  If you are using knitr, include these lines in a code chunk in your document: \\
\begin{Schunk}
\begin{Sinput}
> opts_knit$set(concordance = TRUE)
\end{Sinput}
\end{Schunk}
For Sweave, put these lines outside of a code chunk 
near the start of
your document: \\
\verb!\usepackage{Sweave}!\\
\verb!\SweaveOpts{concordance=TRUE}!\\

The simplest way to proceed is from within R.  Assuming \file{doc.Rnw}
is the knitr/Sweave document to process and it is in the current working directory, run
\begin{Schunk}
\begin{Sinput}
> library(patchDVI)
> knitPDF("doc.Rnw")
\end{Sinput}
\end{Schunk}
or
\begin{Schunk}
\begin{Sinput}
> library(patchDVI)
> SweavePDF("doc.Rnw")
\end{Sinput}
\end{Schunk}
This runs  \file{doc.Rnw} through knitr/Sweave, runs any other chapters in the
project through knitr/Sweave, then runs the main \file{.tex} file (typically
\file{doc.tex}, but not necessarily; see section \ref{sec:make} below)
through \texttt{pdflatex}, and patches the source links.  To produce DVI
output
instead of PDF substitute \texttt{knitDVI} for
\texttt{knitPDF}, and to use \texttt{latex} and \texttt{dvipdfm}
to produce \file{PDF} output, use \texttt{knitDVIPDFM}.
If you are using MikTeX on Windows, the
functions \texttt{knitPDFMiktex} and
\texttt{knitMiktex} correspond to the first two of these respectively,
and use a few MikTeX-specific features.

These functions all have an optional argument \texttt{preview}, which
can contain a command line to run a \file{.pdf} or \file{.dvi} previewer
(with the filename replaced by \verb!%s!).  The \file{.pdf} previewer
should be one that can handle Synctex links; unfortunately, Acrobat
Reader and MacOS Preview are both deficient in this area. On
Windows, \texttt{SumatraPDF} works, as do the built-in previewers
in TeXShop and TeXWorks on MacOS X and other platforms.

MikTeX includes the \texttt{yap}
previewer for \file{.dvi} files; the \texttt{knitMiktex} command sets it
as the default.

Another way to proceed is directly from within your text editor.  The
instructions here depend on your editor; I have included a few in the
Appendices: TeXShop in Appendix \ref{sec:texshop}, WinEdt in
Appendix \ref{sec:winedt}, and TeXWorks in Appendix
\ref{sec:texworks}. Some editors (e.g. TeXShop and TeXWorks)
include a previewer that can handle the source links.

Finally, you may want to run knitr from the command line, outside of R.  This line (or the obvious variants
with replacements for \texttt{knitPDF}) should do it:
\begin{verbatim}
Rscript -e 'patchDVI::knitPDF("doc.Rnw")'
\end{verbatim}




\section{\patchDVI History}\label{sec:history}

Initially  \patchDVI only worked for \file{.dvi} files (hence the name).
It required changes to the Sweave function in R, which 
first appeared around the release of R version
2.5.0. with incompatible changes in R version 2.8.0 when \file{.pdf}
support was added to \patchDVIn.

Using \patchDVI requires a pre-processing step (knitr/Sweave), \LaTeX\ 
processing, and a post-processing step (patching).  This is usually
followed by a preview of the resulting output file.  It quickly became
apparent that it was convenient to package these steps into a
single R function, so the user only needed to make a single call.  But
the details of \LaTeX\  processing vary from platform to platform, so I
wrote functions \texttt{SweaveMiktex} and \texttt{SweavePDFMiktex}
specific to the MikTex platform, with the intention of adding
others as I used them or users told me what needed adding.  This never happened,
but in the meantime, Brian Ripley made the \texttt{tools::texi2dvi} function
in R much more flexible, and in version 1.7 of
\patchDVI I included a modified version of it with the hope that
\patchDVI should be more nearly platform neutral.

The 1.7 release was motivated by an attempt to support TeXWorks \citep{Kew10}, a
cross-platform \LaTeX\ targetted editor.  TeXWorks was still in its early
days (I was working with version 0.2 on Windows), and it did not have
enough flexibility to handle large knitr/Sweave projects, where for example, each
chapter of a book requires separate knitr/Sweave processing, but \LaTeX\ processes
only a main wrapper file.  This prompted me to include more \texttt{make}-style
capabilities into \patchDVIn.  It is now possible to specify a list of
knitr/Sweave input files to process (optionally only if they have changed since the
last processing) and the main wrapper file, all within code chunks
in a single file, using the \texttt{knitAll}/\texttt{SweaveAll} functions.

The \texttt{SweaveDVIPDFM} function is a recent addition.
For English language processing, I find \texttt{pdflatex} to be the most
convenient processor, but it does not work well in languages like 
Japanese.  During a visit to the Institute of Statistical Mathematics
in Tokyo I learned of the issues, and with the help of Prof. H. Okumura
and Junji Nakano I worked out \texttt{SweaveDVIPDFM} to handle the two
step conversion to PDF.

In 2015 I added support for other non-Sweave processors,
such as knitr, and in 2020 improved the documentation
for knitr users.

\section{Sweave Concordances}\label{sec:concordance}

knitr/Sweave processes the code chunks in the \file{.Rnw} file, replacing each
with the requested output from the command.  This means that the output \file{.tex}
file alternates between copied \LaTeX\ source and newly produced blocks of 
output.  Each line in the \file{.tex} file can thus be mapped to 
one or more lines of input, and that is what the concordance does.

\subsection{Technical description of concordance records}

The concordance records are text records in the following format.  There are
four parts, separated by colons:
\begin{enumerate}
\item The label \texttt{concordance} to indicate the type of record.
\item The output \file{.tex} filename.
\item The input \file{.Rnw} filename.
\item The input line numbers corresponding to each output line.
\end{enumerate}
The third component is compressed using a simple encoding:  The first
number is the first line number; the remainder of line numbers are a
run-length encoding of the differences.  Thus if the input file is as
shown in Table \ref{tab:input}, 
\begin{table}
\caption{Input file for simple example.\label{tab:input}}
\begin{center}
\begin{tabular}{ll}
\hline
Line number & Input text \\
\hline
1 & \verb!\SweaveOpts{concordance=TRUE}!  \\
2 & This is text  \\
3 & \verb!<<>>=!  \\ 
4 & \verb!123! \\
5 &  \verb!@! \\
6 & This is more text  \\
\hline
\end{tabular}
\end{center}
\end{table}
the output file would be as
shown in Table \ref{tab:output}, with the concordance as shown there in the second column.
\begin{table}
\caption{Output file for simple example.\label{tab:output}}
\begin{center}
\begin{tabular}{lll}
\hline
Output line & Input line &  Output text \\
\hline
1 & 1 & \verb!\input{sample-concordance}! \\
2 & 2 & This is text. \\
3 & 4 & \verb!\begin{Schunk}! \\
4 & 4 & \verb!\begin{Sinput}! \\
5 & 4 & \verb!> 123! \\
6 & 4 & \verb!\end{Sinput}! \\
7 & 4 & \verb!\begin{Soutput}! \\
8 & 4 & \verb![1] 123! \\
9 & 4 & \verb!\end{Soutput}! \\
10 & 4 & \verb!\end{Schunk}! \\
11 & 6 & This is more text \\
\hline
\end{tabular}
\end{center}
\end{table}
This concordance would be recorded in the file \file{sample-concordance.tex}
as
\begin{verbatim}
\Sconcordance{concordance:sample.tex:sample.Rnw:%
1 1 1 1 2 7 0 1 2}
\end{verbatim}
The numeric part of this file may be interpreted as shown in Table \ref{tab:concordance}.
\begin{table}
\caption{Encoding of numeric part of concordance record.\label{tab:concordance}}
\begin{center}
\begin{tabular}{lll}
\hline
Values & Interpretation & Expansion \\
\hline
\texttt{1} & line 1  & 1 \\
\texttt{1 1} & 1 increase of 1 & 2 \\
\texttt{1 2} & 1 increase of 2 & 4 \\
\texttt{7 0} & 7 increases of 0 & 4 4 4 4 4 4 4 \\
\texttt{1 2} & 1 increase of 2 & 6 \\
\hline
\end{tabular}
\end{center}
\end{table}

\section{Patching \file{.dvi} Files}\label{sec:dvi}

The \verb!\Sconcordance! macro expands to a \verb!\special! macro when
producing a \file{.dvi} file.  This is included verbatim in the \file{.dvi}
file. The ``concordance:'' prefix identifies it as a \patchDVI concordance.
The \texttt{patchDVI} function scans the whole file until it finds
this sort of record.  (There may be more than one, if multiple files make up
the document.)  Source references are also recorded by \LaTeX\ in \verb!\special! records; 
their prefix is ``src:''.  The \texttt{patchDVI} function reads
each ``src:'' special and if it refers to a file in a ``concordance:'' special,
makes the substitution.  At the end, it rewrites the whole \file{.dvi} file.


\section{Patching \file{.synctex} Files}\label{sec:synctex}

When using \texttt{pdflatex}, the \verb!\Sconcordance! macro expands to a
\verb!\pdfobj! macro containing the concordance, which eventually is
embedded in the \file{.pdf} file.  However, the Synctex scheme of
source references does not write the references to the \file{.pdf} file
directly.  Instead, they are written to a separate file with extension
\file{.synctex}, or a compressed version of that file, with
extension \file{.synctex.gz}.  The \texttt{patchSynctex} function
reads the concordances from either the \file{.pdf} file (when \texttt{pdflatex}
was used) or the \file{.dvi} file, and the source
references from the Synctex file.  It rewrites only
the Synctex file when it makes its changes.

\section{Project Management Functions}\label{sec:make}

As mentioned above, there are a number of steps involved in running \patchDVI
with a complex knitr/Sweave project:
\begin{enumerate}
\item Run knitr/Sweave on each input file.
\item Run \LaTeX\ on the main wrapper file.
\item Run the appropriate \patchDVI function on the output file.
\item Preview or print the result.
\end{enumerate}
Moreover, step 1 needs to be repeated once for each \file{.Rnw} file, but only
if the content has changed since the last run, while the other steps need only
be done once.

To manage this complication, the \patchDVI package includes two simple
project management functions, \texttt{knitAll} and \texttt{SweaveAll}.  These are really the same function with different
defaults, and will be described in terms of \texttt{SweaveAll}.  
This function runs Sweave
on multiple files and determines the name of the main wrapper file.  It is 
used internally by the functions described in Section \ref{sec:complete}
below, but can also be called directly by the user.

Here is how it works.  \texttt{SweaveAll} takes a vector
of filenames as input, and runs Sweave on each.  After each run, it examines the
global environment for the four variables \verb!.PostSweaveHook!,
\verb!.SweaveFiles!, \verb!.SweaveMake! and \texttt{.TexRoot}.
(The first three variables can instead be named  \texttt{.PostKnitHook},
\texttt{.knitFiles}, \texttt{.knitMake}.  If both versions are
present, the choice is undefined, so don't do that.)

A code chunk in a \file{.Rnw} file may produce a function (or the name of
a function; \texttt{match.fun} is used to look it up) named \verb!.PostSweaveHook!.
If present, this should be a function taking a single argument.  
Immediately after running \texttt{Sweave}, 
\texttt{SweaveAll} will call this function, passing the name of the 
\file{.tex} output file as the only argument.  The hook can do any
required postprocessing, for example, it could remove local
pathnames from output strings.

The optional parameter \texttt{PostSweaveHook} to the \texttt{SweaveAll} function
can provide a default hook function.  Hooks specified using the \verb!.PostSweaveHook! variable
take precedence in any given input file.

\texttt{SweaveAll} will also use the character variable
named \verb!.SweaveFiles!.  It should contain the names of  \file{.Rnw} 
files in the project.  If no corresponding \file{.tex} file exists, or the
\file{.Rnw} file is newer, they will be run through Sweave.  They may in
turn name additional \file{.Rnw} files to process; each file is processed
only once, even if it is named several times.

There is an optional parameter named \texttt{make} to the \texttt{SweaveAll} function.
If \texttt{make = 1} (the default), things proceed as described above.  If \texttt{make = 0},
the \verb!.SweaveFiles! variable is ignored, and only the explicitly named
files in the call to \texttt{SweaveAll} are processed.  If \texttt{make = 2}, then
all files are processed, whether they are newer than their \file{.tex} 
file or not.  The \verb!.SweaveMake! variable will override the value
of \texttt{make}.  

An \file{.Rnw} file may also set the value of \texttt{.TexRoot} to the name of
a \file{.tex} file.  If it does, then that is the file that should be passed
to \LaTeX\ for processing.  If none is given, then the first file in the call
to \texttt{SweaveAll} will be assumed to be the root file.  (If multiple
different \texttt{.TexRoot} variables are specified by different \file{.Rnw}
files, one of them will be used, but it is hard to predict which:  so don't do
that.)  Whichever file is determined to be the root file is the name returned
by the \texttt{SweaveAll} call.

\texttt{SweaveAll} is called by all of the functions described in subsection
\ref{sec:complete} below to do step 1 of the \patchDVI steps.

The workflow this is designed for is as follows.  Each \file{.Rnw}
chapter (named for example ``chapter.Rnw'') 
in a large project should specify the \texttt{.TexRoot}, e.g. using the
code chunk
\begin{verbatim}
 <<echo=FALSE>>=
 .TexRoot <- "wrapper.tex"
 @
\end{verbatim}
Similarly, the wrapper file (named for example
``wrapper.Rnw'') should be a \file{.Rnw} file that sets
\verb!.SweaveFiles! to the complete list of files in the project.
Then one can build an initial copy of the entire document by calling
any of \texttt{knitPDF}, \texttt{SweavePDF}, \texttt{knitMiktex}, \texttt{SweaveMiktex}, \texttt{knitDVI}, \texttt{SweaveDVI}, \texttt{knitDVIPDFM}, \texttt{SweaveDVIPDFM} 
with argument \texttt{"wrapper.Rnw"}.
Later, while one is working on \texttt{"chapter.Rnw"}, one can call
one of those functions
with argument \texttt{"chapter.Rnw"} 
and the chapter will be processed through
the full sequence, without running knitr/Sweave on the other chapters.

More complicated schemes are possible.  For example:
\begin{itemize}
\item Each chapter can have subsections in separate
files; then the chapter would name the subsections, but the main wrapper would only
need to name the chapters if you can assume that only the chapter
being edited was changed.
\item If one wants to ``make'' the full project every time,
then include \texttt{"wrapper.Rnw"} in \verb!.SweaveFiles! in each chapter.
\end{itemize}

\subsection{The Complete Process}\label{sec:complete}

The \patchDVI package contains five functions for each
of knitr and Sweave designed to run all
four of the steps listed at the start of this section.  The functions
\texttt{knitDVI}/\texttt{SweaveDVI} and \texttt{knitMiktex}/\texttt{SweaveMiktex} produce \file{.dvi} output
in the general case and for MikTeX respectively; \texttt{knitPDF}/\texttt{SweavePDF} and
\texttt{knitPDFMiktex}/\texttt{SweavePDFMiktex} do the same for direct \file{.pdf} output
from \texttt{pdflatex}.  Finally, \texttt{knitDVIPDFM}/\texttt{SweaveDVIPDFM} run
the two-step conversion using first \texttt{latex} and then \texttt{dvipdfm}.

In each case, the \TeX\ processing functions are customizable.  

For example, a few years ago I had a text editor that allowed me to call external
functions with arguments depending on the name of the current file and
the line number within it.  I had it call a Windows batch file 
with the line set as argument \verb!%1! and the filename
set as argument \verb!%2!; the batch
file invoked R using the command line
\begin{verbatim}
echo patchDVI::SweaveMiktex('%2', 
    preview='yap -1 -s"%1%2" "\x25s"') 
    | Rterm --slave
\end{verbatim}
(all on one long line).  This passed the current file to \texttt{SweaveMiktex},
and set the preview command to use the \texttt{yap} options \texttt{-1} to 
update the current view (rather than opening a new window), and to jump to the
line corresponding to the editor line.  The code \verb!"\x25s"! is simply
\verb!"%s"! encoded without an explicit percent sign, which would be 
misinterpreted by the Windows command processor.  When \patchDVI calls
the previewer, the main \file{.dvi} filename will be substituted for
\verb!%s!.


\subsection{Installing or Loading Packages}

In a complex project, there are often a number of different packages
required.  When updating R, you may end up with a tedious exercise to
make sure these are all installed and updated.

The \texttt{needsPackages()} function helps with this.  It takes a
character vector naming packages that will be used in the current 
document.  By default, it installs any that are not already installed.
Optionally, it can update them using \texttt{update.packages()}, 
load them, or attach them to the search list.  For example, this
document uses no packages other than \texttt{patchDVI} itself, 
so it could have 
\begin{Schunk}
\begin{Sinput}
> patchDVI::needsPackages("patchDVI")
\end{Sinput}
\end{Schunk}
\noindent
near the start to ensure it is available, if this wasn't a
nonsensical statement.  (Why would you be able to run it if \texttt{patchDVI} wasn't already installed?)


\section{Conclusion}\label{sec:conclusion}

As described in this paper, the \patchDVI package is a convenient way
to work with knitr/Sweave in a modern setting, allowing fast switching from
source input to preview.  It also offers some features to make the
management of larger projects easier.  

Other possibilities may exist to make use of the code in this package.  In
order to read and patch \file{.dvi}, \file{.pdf} and \file{.synctex}
files, \patchDVI includes code to work with each of those formats.  Users may
find imaginative uses for this capability, which I've tried to leave 
in general form.  The low-level \file{.dvi} editing is done by C functions
called from R, while the PDF related work is done in pure R code.  

\bibliography{patchDVI}

\newpage

\appendix

\section{Using \patchDVI with TeXShop}
\label{sec:texshop}

TeXShop is a nice \TeX\ editor on MacOS.  Dave Gabrielson of the University of Manitoba helped me to work
out these instructions.  I have updated them in December, 2013 for TeXShop 2.47.

\begin{enumerate}
\item In Preferences -- Typesetting -- Sync Method, choose ``SyncTeX''.
\item 
\begin{enumerate}
\item To use with Sweave, create a file called \\
\verb!Library/TeXShop/Engines/Sweave.engine!  \\
containing the lines
\begin{verbatim}
#!/bin/bash
export LC_ALL=<locale>
Rscript -e "patchDVI::SweavePDF('$1')"
\end{verbatim}
in your home directory, and give it executable permissions.  Replace \verb!<locale>! with
your locale string, e.g. \verb!en_CA.UTF-8! for Canadian English using UTF-8 encoding.  The
locale line can be omitted if you only use plain ASCII characters, but is probably necessary
for other cases.
\item To use with knitr, create a file called \\
\verb!Library/TeXShop/Engines/knitr.engine! \\
containing the lines
\begin{verbatim}
#!/bin/bash
export LC_ALL=<locale>
Rscript -e "patchDVI::knitPDF('$1',\ 
  envir = globalenv())"
\end{verbatim}
in your home directory, and give it executable permissions.  Replace \verb!<locale>! with
your locale string, e.g. \verb!en_CA.UTF-8! for Canadian English using UTF-8 encoding.  The
locale line can be omitted if you only use plain ASCII characters, but is probably necessary
for other cases.
\item For other vignette engines, use a \verb!weave! argument in the above, as appropriate.
\end{enumerate}
\item Install the \texttt{patchDVI} package into R.
\item When editing a \texttt{.Rnw} file in TeXShop, choose the knitr or Sweave engine from the menu.
\item If you have multiple files in your project, your main file must be a \file{.Rnw} file
(e.g. \texttt{Main.Rnw}) which lists all \file{.Rnw} files in a \verb!.SweaveFiles! variable, 
and you need to add the line
\begin{verbatim}
%!TEX root = Main.Rnw
\end{verbatim}
to each subordinate file.
\item For Sweave, add the \verb!\SweaveOpts{concordance=TRUE}! line to your document.  For knitr, add a code chunk
similar to this:
\begin{verbatim}
 <<results='asis'>>=
 patchDVI::useknitr()
 @
\end{verbatim}
somewhere near the start of your document.
\end{enumerate}

The TeXShop previewer supports SyncTeX; you right click in the preview, and choose Sync from the menu
to jump to your source location.

\section{Using \patchDVI with TeXWorks}
\label{sec:texworks}

TeXWorks is an editor for multiple platforms, 
somewhat similar to TeXShop.  These instructions 
have been tested in version 0.4.5, with MikTeX 2.9 on Windows,
and version 0.6.2 from MacTeX on MacOS.  

NB:  Some versions of TeXWorks had a bug in setting the \texttt{HOME} directory
of the user.  With those versions, R will not find a locally installed
copy of \texttt{patchDVI}.  To work around the bug, set the \verb!R_USER!
environment variable to your Windows home directory, e.g. \verb!R_USER=C:/Users/Murdoch!.

TeXWorks can work with the \texttt{patchDVI}
project management features using a script
to tell it to process the current file through knitr/Sweave, but preview
the main file.  See the instructions below for my current best attempt at such
a script.  It can also use the TeXShop approach of specifying
the TEX root file to be a \file{.Rnw} file.

The instructions are given first for Sweave, then below for knitr.

\begin{enumerate}
\item \label{theCommand} Add a new SweavePDF command:  In\\

Edit | Preferences | Typesetting \\
\\
click on the ``+'' sign near the bottom.  Set the name of the tool to be SweavePDF.
Set the program to Rscript.

Add two arguments, one per line:
\begin{enumerate}
\item \verb!-e!
\item \verb!patchDVI::SweavePDF('$fullname')! % $
\end{enumerate} 
\item Install the \texttt{patchDVI} package into R.
\item Tell TeXWorks to open Sweave files by editing the file pattern configuration file \texttt{texworks-config.txt}.
This file is in the \texttt{configuration} folder of the TeXWorks home directory.  For example, I have this
line in my file:
\begin{verbatim}
file-open-filter:	Sweave and TeX documents (*.Rnw *.tex)
\end{verbatim}
\item When editing a \texttt{.Rnw} file in TeXWorks, choose the SweavePDF engine from the menu.
\item \label{theConcordance} Add the \verb!\SweaveOpts{concordance=TRUE}! line to your document.
\item If you are using the project management features of \texttt{patchDVI} and are
editing a subordinate file, TeXWorks will not open or update the PDF preview after it processes changes.
There are four workarounds for this.

The simplest is to manually open the \texttt{.pdf} file the first time.  After that it will be updated
automatically.  Unfortunately, if you happen to be editing the main file, the \texttt{.pdf} will be opened
automatically, and then updates won't happen if you later edit a subordinate file.

The next simplest is the TeXShop approach:  include a line 
\begin{verbatim}
%!TEX root = Main.Rnw
\end{verbatim}
near the top of the file, and make sure that \texttt{Main.Rnw} refers
to all subordinate Sweave files.

\end{enumerate}


To use TeXWorks with knitr, the instructions are very similar to those above, but with two changes.

In step \ref{theCommand}, replace the second line of the command (the \verb!SweavePDF! call) with the following longer command, all on one line:
\begin{verbatim}
patchDVI::SweavePDF('$fullname', weave = knitr::knit, 
envir = globalenv())
\end{verbatim}

In step \ref{theConcordance}, insert the following code chunk
into your file:
\begin{verbatim}
 <<results='asis'>>=
 patchDVI::useknitr()
 @
\end{verbatim}

\section{Using \patchDVI with WinEdt}
\label{sec:winedt}

WinEdt is a Windows editor with \TeX\ support.  The configuration options have changed a number of times; I do
not know how to implement these instructions in the latest version.  These instructions apply to version 5.5,
and assume you are using it with MikTeX.

\begin{enumerate}
\item In Options -- Execution Modes choose Texify, and click on Browse for Executable.  Find the \texttt{Rscript} executable
in your R installation, directory \file{bin/i386} or \file{bin/x64}, and choose it.  In the Switches line, put 
\begin{verbatim}
-e
\end{verbatim}
and in the Parameters line, put
\begin{verbatim}
"patchDVI::SweaveMiktex('%n%t', '%N.tex')"
\end{verbatim}
The quotes are necessary!
\item Do the same for the PDF Texify command, replacing \texttt{SweaveMiktex} with \texttt{SweavePDFMiktex}.

\item In Options -- Execution modes, make sure Start Viewer and Forward Search are selected for LaTeX and PDF LaTeX.
\end{enumerate}

When you preview a file in \texttt{yap}, double clicking should jump back to the editor.  If it doesn't (or it opens the wrong editor),
while you're in \texttt{yap} choose View -- Options -- Inverse DVI search.  
You should see ``WinEdt (auto-detected)" as an option; if so, select it.  If not, create a new entry for WinEdt, and for the command line, put in
\begin{verbatim}
"path\to\winedt.exe" "[Open(|%f|);SelPar(%l,8)]"
\end{verbatim}
after editing the path as necessary.

\section{Using \patchDVI with RStudio}
\label{sec:rstudio}

RStudio is a very nice front end for working in R and with 
individual \file{.Rnw} or Markdown files.  If you are using
it, I'm going to assume you're using knitr as well, and
these instructions have been worked out for that combination.

RStudio is less flexible than the other editors for specifying
customized processing of a file, so these instructions were 
worked out assuming that you already have it configured for 
knitr.  It is probably possible to do something similar for Sweave;
I just haven't tried.

You need to set up your individual chapter files as for
TeXShop/TeXWorks, i.e. with a 
\begin{verbatim}
%!TEX root = Main.Rnw
\end{verbatim}
comment at the top of each.  This tells RStudio to run knitr
on the main file when you click Compile PDF.  (It will also 
work if you use the knitr style
\begin{verbatim}
%!RNW root = Main.Rnw
\end{verbatim}
but then your files won't work in TeXShop/TeXWorks.)

In the main file, you need a code chunk containing
a line to set the \verb!.SweaveFiles! variable naming
all chapter files (but not the main file), and then running \texttt{knitInRStudio}:
\begin{Schunk}
\begin{Sinput}
> .SweaveFiles <- c("a.Rnw", "b.Rnw")
> patchDVI::knitInRStudio()
\end{Sinput}
\end{Schunk}
It is safe to put these lines in your file even if you
sometimes process it in a different way:  if you are
not in RStudio, \texttt{knitInRStudio} does nothing.

One remaining issue with this approach is that you won't see 
the knitr progress messages from knitting the chapter files.
If you want to see those messages, add the chunk option
\texttt{childOutput = TRUE} to the code chunk holding this
code.

 

\end{document}




