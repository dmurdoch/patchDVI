%!TEX encoding = UTF-8 Unicode
%!TEX TS-program = JSweave
%!TEX program = JSweave

% This document needs the Japanese localized version of LaTeX, uplatex,
% as described in the text in section 1.

\documentclass[12pt]{ujarticle}
\usepackage{Sweave}
\usepackage{natbib}
\newcommand{\ThisIsNotUsed}{
\usepackage[utf8]{inputenc}
}

\bibliographystyle{plain}

\Sconcordance{concordance:Japanese.tex:Japanese.Rnw:%
1 239 1}


\newcommand{\patchDVI}{\patchDVIn\ }
\newcommand{\patchDVIn}{\textbf{patchDVI}}
\newcommand{\file}[1]{\texttt{#1}}

%\VignetteEngine{patchDVI::JSweave}
%\VignetteIndexEntry{Using Sweave and patchDVI with Japanese text}
%\VignetteKeyword{Sweave}
%\VignetteKeyword{pdf}

\title{Using Sweave and \patchDVI with Japanese text}
\author{Duncan Murdoch}

\begin{document}

\maketitle

\abstract{
The \patchDVI package works with Sweave \citep{Leisch02}  and
document previewers to facilitate editing:  it modifies the links
that \LaTeX\ puts into the output so that they refer to the original
source.  It also includes a few project management functions to 
make large multi-file Sweave documents easier to handle.  This
document describes Japanese language support in the
package.}

\newpage

\tableofcontents

\newpage

\section{Introduction}

The main vignette ``The patchDVI package'' describes general features
of patchDVI, and how to set up various editors to use it in typesetting
English-language vignettes.  This vignette describes how to set it up
for Japanese language text.  I expect the issues would be the same
in other non-Roman alphabets, but I have no experience with them.

The main issues in writing Japanese language (日本語 ) documents are the
encoding of the 
character set and the varying support by different \LaTeX\ engines.  The 
standard \texttt{pdflatex} engine does not work well with Japanese. 
In November, 2013 I visited the Institute for Statistical Mathematics
in Tokyo, and with the help of Mr. Yasuto Nakano and Prof. Haruhiko Okumura
I developed these instructions.  In outline:  you must use the UTF-8 encoding
for the characters, use \texttt{uplatex}, \texttt{platex} or \texttt{xelatex} 
to convert to a DVI file, and use
\texttt{dvipdfmx} to convert the DVI to PDF.  These instructions
describe the procedure using \texttt{uplatex};  I expect modifications
for the other engines will be minor. 

\section{Creating a Vignette in an R Package}

To create a ``vignette'' in an R package that contains Japanese language text, follow these 
steps.

\begin{enumerate}
\item In the \texttt{DESCRIPTION} file of the package, add the lines
\begin{verbatim}
Suggests:  patchDVI
VignetteBuilder:  patchDVI
\end{verbatim}


\item Put the Sweave file containing the source for the vignette in the \texttt{vignettes}
directory of the package.  

\item As with all vignettes, it needs a line like 
\begin{verbatim}
  %\VignetteIndexEntry{Using Sweave and patchDVI with Japanese text}
\end{verbatim}
somewhere near the beginning, to give the text that will be shown by \verb!browseVignettes()!
and similar R functions.

\item To process it through \texttt{uplatex} and \texttt{dvipdfmx} instead of the default
engines, add the line
\begin{verbatim}
  %\VignetteEngine{patchDVI::JSweave}
\end{verbatim}
near the beginning.
\end{enumerate}

\section{Using \patchDVI with TeXShop}
\label{sec:texshop}

These instructions have been tested with TeXShop version 2.47 and 3.36.1, but should work with 
other versions as well.

\begin{enumerate}
\item In Preferences -- Typesetting -- Sync Method, choose ``SyncTeX''.
\item Create a file called \verb!Library/TeXShop/Engines/JSweave.engine! 
containing the two lines
\begin{verbatim}
#!/bin/sh
\end{verbatim}
and
\begin{verbatim}
Rscript -e "Sys.setlocale(locale='en_US.UTF-8');
patchDVI::SweaveDVIPDFM(  '$1', latex='uplatex', 
dvipdfm='dvipdfmx', encoding = 'UTF-8' )"
\end{verbatim}
(which I have split to fit in this document; the actual line should not be split).
Give this file executable permissions.  
\item Install the \texttt{patchDVI} package into R.  For this to work,  you need
at least version 1.9.1587 of patchDVI.
\item Add the lines 
\begin{verbatim}
%!TEX encoding = UTF-8 Unicode
%!TEX TS-program = JSweave
\end{verbatim}
at the beginning of all files, and make sure they are created with UTF-8 encoding.
\item If you have multiple files in your project, your main file must be a Sweave file
(e.g. \texttt{Main.Rnw}) which lists all Sweave files in a \texttt{.SweaveFiles} variable, 
and you need to add the line
\begin{verbatim}
%!TEX root = Main.Rnw
\end{verbatim}
to each subordinate file.
\item Add
\begin{verbatim}
\usepackage[utf8]{inputenc}
\end{verbatim}
to the main file to declare to Sweave that it uses the UTF-8 encoding.  Add
\begin{verbatim}
\newcommand{\ThisIsNotUsed}{
\usepackage[utf8]{inputenc}
}
\end{verbatim}
to subordinate files.  This defines a macro which is never used; the purpose is
for Sweave to see the \texttt{inputenc} line.  (You can also use this line in the main file
if you are using a class file such as \texttt{ujarticle} which defines the encoding, 
and don't want \texttt{inputenc} to be used.)
\item \label{point:eps} Add the lines
\begin{verbatim}
 \usepackage{Sweave}
 \SweaveOpts {concordance=TRUE,eps=TRUE,pdf=FALSE}
\end{verbatim}
near the start of each file.  In subordinate files, the first line should be commented out, e.g.
\begin{verbatim}
 % \usepackage{Sweave}
 \SweaveOpts {concordance=TRUE,eps=TRUE,pdf=FALSE}
\end{verbatim}
The reason for this is that Sweave generates \verb!\inputgraphics! calls with no extensions on the files,
and the \texttt{uplatex} processor will not find PDF files unless they are explicitly 
named.  (Advanced users will know other ways to handle this issue.)

\end{enumerate}

With these changes, the TeXShop previewer will support SyncTeX; you right click in the preview, and choose Sync from the menu
to jump to your source location.

\section{Using \patchDVI with TeXWorks}
\label{sec:texworks}

TeXWorks is an editor for several platforms, somewhat similar to TeXShop.  These instructions 
have been tested in TeXWorks 0.4.4 on MacOS; I have not tested them on Windows, because
I do not have \texttt{uplatex} and \texttt{dvipdfmx} there, but they should work if those
programs are available. 

\begin{enumerate}
\item 
In 
Edit -- Preferences -- Typesetting 
click on the ``+'' sign near the bottom of the Processing Tools box.  Set the name of the tool to be JSweave.
Set the program to Rscript.  (You may need to give the fully qualified path
if R is not in your system PATH.)

Add two arguments on separate lines.  The first is simply -e, the second is
\begin{verbatim}
{Sys.setlocale(locale='en_US.UTF-8');
 patchDVI::SweaveDVIPDFM(  '$fullname',
 latex='uplatex', dvipdfm='dvipdfmx', 
 encoding = 'UTF-8' )}
\end{verbatim}
This should be typed all on one line.

\item Install the \texttt{patchDVI} package into R.  For this to work,  you need
at least version 1.9.1587 of patchDVI.
\item Add the lines 
\begin{verbatim}
%!TEX encoding = UTF-8 Unicode
%!TEX program = JSweave
\end{verbatim}
at the beginning of all files, and make sure they are created with UTF-8 encoding.
\item If you have multiple files in your project, your main file must be a Sweave file
(e.g. \texttt{Main.Rnw}) which lists all Sweave files in a \texttt{.SweaveFiles} variable, 
and you need to add the line
\begin{verbatim}
%!TEX root = Main.Rnw
\end{verbatim}
to each subordinate file.
\item Add
\begin{verbatim}
\usepackage[utf8]{inputenc}
\end{verbatim}
to the main file to declare to Sweave that it uses the UTF-8 encoding.  Add
\begin{verbatim}
\newcommand{\ThisIsNotUsed}{
\usepackage[utf8]{inputenc}
}
\end{verbatim}
to subordinate files.  This defines a macro which is never used; the purpose is
for Sweave to see the \texttt{inputenc} line.  (You can also use this line in the main file
if you are using a class file such as \texttt{ujarticle} which defines the encoding, 
and don't want \texttt{inputenc} to be used.)

\item Add 
\begin{verbatim}
 \SweaveOpts {concordance=TRUE,eps=TRUE,pdf=FALSE}
\end{verbatim}
See the note in Section \ref{sec:texshop} point \ref{point:eps} for a discussion of this.
\end{enumerate}

The TeXWorks previewer will jump back to the source if you right click and choose Jump to Source.

\bibliography{patchDVI}
\end{document}




